\chapter{Introduction}
The earliest developers of Artificial Intelligence were fascinated by having smart agents that could find better solutions than humans, in a possibly shorter time. However, when this approach started to be applied to gaming agents, it eventually led to frustrating opponents\cite{soni2008bots}, due to it being too-challenging of a game, or a too-easy and boring one. At the same time, games' developers found themselves in need of offering more realistic content, with interesting and authentic-looking nonplayer characters (NPC)\cite{togelius2011search}, and automated and reliable game testers\cite{holmgaard2015monte}. To solve this problem, and make gaming a more enjoyable experience, research has started to develop interest in Artificial Intelligence behaviour, trying to mimic humans' game playing, mostly through Imitation Learning - the meaning of which is straightforward. To quote Gorman\cite{gorman2006believability}, Imitation Learning is "the acquisition of skills or
behaviors through examination of a demonstrator’s execution of a given task". Few different strategies have been applied so far, often associated to Artificial Neural Networks (ANN), as in \cite{soni2008bots,cho2006exploiting,zanetti2004machine,spronck2003improving,mirandaneuroevolution}, but also Reinforcement Learning algorithms\cite{spronck2003online}, or Genetic Algorithms\cite{martinez2016creating}, among others.\\
Monte Carlo Tree Search has been the object of focus in the process to "humanize" artificial agents, applied on Spades\cite{devlin2016combining}, play testing\cite{holmgaard2015monte}, or video games\cite{khalifamodifying}, for example. However, we find that the research could be exploited further with this thesis project, which merges several different tools to model \emph{procedural personæ}, defined by Holmgård as: \say{Game playing agents that codify player decision making styles, either from the designer's holistic representation of these or from observations of players collected directly from the game}\cite{holmgaard2016procedural}. In this thesis, we will use the terms personality and procedural persona as interchangeable.\\
This thesis work has mostly been carried out at Reykjavík University, on the basis of an existing project that was conjoining virtual reality and general game playing\cite{helgadottir2016virtual}. The agent developed was a graphical representation of a version of CadiaPlayer\cite{cadiaplayer}, a general game player implemented by the Reykjavík University team that has three times won the international General Game Playing competition\cite{genesereth2013international}. 
\section{Goal of the project}
This project's goal is to deduce if personality is a feature that can be inferred from game playing. Aiming to do so, information about human played games have been collected and used to train a Genetic Algorithm over a Monte Carlo Tree Search algorithm applied to the data gathered. The search algorithm would be returning a move selection for the current state of the game, and the Genetic Algorithm would return the set of parameters that would make the search match the same moves as the human players. The outputted parameters will then be evaluated over a new set of human-played matches, confirming or disproving the hypothesis behind this thesis.\\
In the case that the results will be satisfactory, it can lead to the development of more targeted agents, not necessarily only in the game playing field.
\section{Structure of the thesis}
The structure of this thesis is as follows: firstly, an introductory chapter with an overview on the relation between games and artificial intelligence, in addition of a general description of the problem. The following chapter is a survey on the theoretical notions necessary to understand the project and its implementation, starting with a review of the personality models we came across; the definition of General Game Playing; Monte Carlo Tree Search, as the algorithm we adopt in our General Game Player; an overview on Machine Learning; and, lastly, the definition of Genetic Algorithms. The third chapter then describes the methodologies applied at implementation time, justifying the choices made during the work process. Sequentially, we redefine the problem; describe the data collection process and the base project used as a starting point; define the Monte Carlo Tree Search variations adopted; formalize the probability model; specify the Genetic Algorithm's options selected; and outline the evaluation process adopted. In the fourth chapter we include the results of each step, followed by an explicative discussion over the outcomes at various stages of the project: the data collection, the output of the genetic algorithm, and, finally, the results of the game playing evaluation. Lastly, chapter five includes the conclusions drawn during this thesis work, together with a brief overview on how this project could be expanded in the future.