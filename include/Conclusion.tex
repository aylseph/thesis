\chapter{Conclusion}
In order to infer a personality from human played games, a genetic algorithm has been trained with the aid of collected human-played games. The data has been gathered through the Tiltyard server\cite{tiltyard}, which is a platform for General Game Playing where to test and develop General Game Players. General Game Playing is also the framework used in this project (see section \ref{sec:ggp}), where no knowledge of the game is priorly required. This encouraged having a relatively wide selection of games, allowing the users to select any among the games of Checkers, Connect Four, Nine-Boards Tic-Tac-Toe, and Skirmish. The users also provided information about their personality, based on the Galen-Hippocrates' Four Temperaments model (see section \ref{sec:hippocrates}), which has purposely been chosen as model for this thesis due to its simplicity. The games collected have then been partially used to feed a genetic algorithm (see section \ref{sec:gatheory}), where each individual represented a set of parameters for a Monte Carlo Tree Search algorithm (see section \ref{sec:mctstheory}). The Monte Carlo Tree Search is the main algorithm behind the artificial player used in this thesis, and has been supplied with a set of enhancements, each of which is depending on a control variable. The set of those control variables is the aforementioned individual from the genetic algorithm. Once the GA has been running on a set of game representing a specific personality, the fittest individual is selected for evaluation \ref{sec:meteval}. This reduced to 4 different sets of parameters, one for each personality in the model, that have been plugged in the artificial player, and checked with games that have been left out from the training process.\\
The evaluation process consisted in calculating the Action Agreement Ratio (AAR), together with confronting the QValues of the chosen moves as $\Delta_Q$. The results of this evaluation, collected in chapter \ref{sec:results}, proved that many search-selected moves have an imperceptible $\Delta_Q$, whilst still not reaching perfect matching. This could mean that the gaming style achieved is comparable to human's, although not taking the exact same decisions. However, it is still to be confirmed if it is personality biased.\\
The results obtained in this project are promising enough to believe that a probability model could be inferred from games data. Although the dataset collected and utilised as testbed was not adequate to be considered fully statistically relevant, therefore not confirming nor disproving the thesis where a personality model can be inferred from games data, we are confident that a final conclusion can be drawn with further research.\\
Additional focus would be applicable in determining the best MCTS enhancements and how they affect the search; tweaking the GA parameter so to find a better environment for games data training; applying a different personality model that could be better representative of the players' behaviours; and implement some machine learning strategies to capture patterns in actions' selections for different players. However, none of the above would have meaningful results if the dataset would not be remarkably increased.
