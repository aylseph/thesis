Modeling Players Personality in General Game Playing\\
Stefania Crotti\\
Department of Computer Science and Engineering\\
Chalmers University of Technology and University of Gothenburg\setlength{\parskip}{0.5cm}

\thispagestyle{plain}			% Suppress header 
\setlength{\parskip}{0pt plus 1.0pt}
\section*{Abstract}
Artificial agents’ skills need to become more relatable to humans’, and one approach to solve this problem would be to associate a personality to the agents. When games are used as a framework, General Game Playing (GGP) provides an unbiased environment where new games are played without any prior knowledge of the rules, and without applying any game-dependent heuristic. This thesis is expecting to infer preferences from human played games, depending on the personality the players recognised themselves in. The artificial player is aided with a Monte Carlo Tree Search algorithm with tunable parameters, which associate evaluation values to each move, consequently selecting the next state. The optimal set of parameters to fit the human gameplay is found with the subsidy of a Genetic Algorithm where individuals are represented as sets of parameters themselves. This approach is backed up with a Bayesian probability model, and, finally, the outputted sets of parameters are evaluated to determine if the artificial gamer has indeed learnt to behave accordingly to a certain personality. After an extensive research on personality models has been carried out to find a suitable one for the amount of data expected to be collected, the choice has fallen over the Hippocrates’-Galen Four Temperaments. The results however hint to the conclusion that a different model might have been easier to be fit. Although the results are not astonishing, this thesis can be considered as a first stepping stone into personality model fitting through Monte Carlo Tree Search parameters tuning.

% KEYWORDS (MAXIMUM 10 WORDS)
\vfill
Keywords: General Game Playing, Monte Carlo Tree Search, Genetic Algorithm, Personality Mapping, Bayesian Modeling.

\newpage				% Create empty back of side
\thispagestyle{empty}
\mbox{}